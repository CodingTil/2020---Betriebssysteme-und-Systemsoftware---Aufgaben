\documentclass[11pt]{article}

%Packages
\usepackage{amsfonts}	      %Mathematische Zeichen und Fonts
\usepackage{mathtools}        %Extra Mathematische Symbole
\usepackage{extarrows}	      %Extra Pfeile
\usepackage{listings}         %Codeansicht
\usepackage{scrlayer-scrpage} %Seitenkopf
\usepackage{tikz}             %tikz
\usepackage{enumitem}		  %Enumerate
\usepackage{listings}		  %Code snippets
\usepackage{amsmath}

\usetikzlibrary{arrows, automata, positioning}
\pagestyle{scrheadings}

\begin{document}

%Header
\ihead{\textbf{Betriebssysteme und Systemsoftware \\ Übungsblatt 3} \\Tutorium 1}
\ohead{Andrés Montoya, 405409 \\ Til Mohr, 405959\\ Marc Ludevid Wulf, 405401}

Vorlesungen 6.2.c - 6.2.e sind immer noch nicht auf Moodle, 2 Tage vor der Abgabe....

%Seiteninhalt
\paragraph{Aufgabe 5.1}


\paragraph{Aufgabe 5.2}
\begin{enumerate}[label=\alph*)]
\item Sei \verb|use| ein Zählsemaphore, der auf den Wert \verb|0| initialisiert wurde. Seien \verb|empty| und \verb|full| Zählsemaphore, die auf \verb|n| und \verb|0| initialisiert werden und beide Maximalwerte von \verb|n| haben.
\begin{lstlisting}[language=C]
void enqueue(element) {
	wait(empty);
	wait(use);
	queue.add(element);
	signal(use);
	signal(full);
}
\end{lstlisting}

\begin{lstlisting}[language=C]
element dequeue() {
	wait(full);
	wait(use);
	element := queue.pop();
	signal(use);
	signal(empty);
	return element;
}
\end{lstlisting}

\item Es können zwei Wagen gleichzeitig einfahren, falls 2 Wagen aufs einfahren Warten (\verb|while|-Schleife Zeile 5), \verb|eingefahreneneWagen| auf \verb|0| gesetzt wird, dann zuerst der eine Wagen die \verb|while|-Schleife verlässt, dann der andere Wagen, bevor der erste Wagen überhaupt \verb|eingefahreneneWagen| auf \verb|1| setzen konnte.
\\\\Es können mehrere Besucher gleichzeitig einen Wagen betreten: Sei 1 Wagen gerade eingefahren, 3 Besucher warten aufs Betreten (\verb|while|-Schleife Zeile 17). Wenn wie oben alle 3 Besucher direkt hintereinander deren \verb|while|-Schleifen verlassen, ohne Zeile 19 einmal auszuführen, können diese 3 Besucher den Wagen betreten.
\\\\In beiden Szenarien sind die Anzahlen nur beispiele, es kann für beliebiges $n\in\mathbb{N}$ auftreten.
\\\\Zudem ist wird nie sichergestellt, dass sowohl Besucher als auch Wagen der Reihe nach betreten/einfahren. Gerade bei den Wagen kann dies fatal werden, wenn sie auf einer festen Spur fahren und ein hinterer Wagen einfahren will und dabei andere Wagen zerstört.

\item 

\item

\end{enumerate}


\paragraph{Aufgabe 5.3}

\end{document}